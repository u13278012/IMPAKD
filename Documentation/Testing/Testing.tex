\documentclass[a4paper,12pt]{article}
\addtolength{\oddsidemargin}{-1.cm}
\addtolength{\textwidth}{2cm}
\addtolength{\topmargin}{-2cm}
\addtolength{\textheight}{3.5cm}
\newcommand{\HRule}{\rule{\linewidth}{0.5mm}}
\makeindex

\usepackage{titlesec}
\usepackage{longtable}
\usepackage[pdftex]{graphicx}
\usepackage{makeidx}
\usepackage{hyperref}
\usepackage{verbatim}
\hypersetup{
    colorlinks=true,
    linkcolor=blue,
    filecolor=magenta,      
    urlcolor=cyan,
}


% define the title
\author{IMAPKD}
\title{ Testing}
\begin{document}
\setlength{\parskip}{6pt}

% generates the title
\begin{titlepage}

\begin{center}
% Upper part of the page       
\includegraphics[width=1\textwidth]{./University_of_Pretoria_Logo.PNG}\\[0.4cm]  

%Title

\HRule \\[0.4cm]
{\LARGE IMPAKD}\\[0.8cm]
{\LARGE Project $\>$ $\>$   COS 301 Main Project}\\[0.5cm]
{\LARGE Product $\>$ $\>$  Test Plan and Report}\\[0.8cm]
{\LARGE Version $\>$ $\>$  1.2}\\
\HRule \\[4cm]  

% Group Members 

%\begin{minipage}{0.4\textwidth}
%\begin{flushleft} \large

%\emph{\Large Group Members:}\\[0.4cm]    
%\emph{}\\
%{\Large Diana {Obo}} \\
%\emph{}\\
%\emph{}\\
%{\Large Priscilla {Madigoe}}\\
%\emph{}\\
%\emph{}\\
%{\Large Kudzai {Muranga}} \\
%\emph{}\\
%\emph{}\\
%{\Large Sandile {Khumalo}}\\
%\emph{}\\
%\emph{}\\

%\end{flushleft}
%\end{minipage}
%\begin{minipage}{0.4\textwidth}
%\begin{flushright} \large

%\emph{ \Large Student numbers:} \\[0.4cm]  
%\emph{}\\
%{\Large u13134885}\\
%\emph{}\\
%\emph{}\\
%{\Large u13049128}\\
%\emph{}\\
%\emph{}\\
%{\Large u13278012}\\
%\emph{}\\
%\emph{}\\
%{\Large u12031748}\\
%\emph{}\\
%\emph{}\\

%\end{flushright}
%\end{minipage}

{ \huge \bfseries Unit Test Plan and Report}\\[4cm]

% bottom right section of title page

\hfill{\large Prepared by:}\\[0.5cm]
\hfill{\large Sandile Khumalo - 12031748}\\[0.1cm]
\hfill{\large Kudzai Muranga - 13278012}\\[0.1cm]
\hfill{\large Diana Obo - 13134885}\\[0.1cm]
\hfill{\large Priscilla Madigoe - 13049128}\\[0.5cm]
\hfill{\large \today}

\end{center}
\end{titlepage}
\renewcommand{\thesection}{\arabic{section}}

\newpage
\begin{center}
\textsc{\Large IMPAKD link}\\[0.5cm]
For further references see \href{https://github.com/u13278012/IMPAKD/}{gitHub}.
\today
\end{center}
\newpage
\tableofcontents{}

\setcounter{secnumdepth}{4}

\newpage
\section{Introduction}
\subsection{Purpose}
The Property Investment Optimiser's purpose is to assist the user by optimising their investments in the rental property space. The purpose of this document is to list the tests we have put in place to ensure the functionality of the project. Test driven development was used to ensure that we spend less time debugging the code while creating a detailed specification of what functionality is required for the project.

\subsection{Scope}
This document will be used to elaborate and substantiate the unit tests conducted for the various use
cases of the Property Investment Optimiser project. Unit test, with mock objects, will be used to ensure the correct functionality of each use case's implementation. These unit tests are done on the server side.

\subsection{Test Environment}
Junit was used to implement the tests for each test case. Also, EJBContainers were used to test functionality that requires JPA and Enterprise Java Beans.

%\subsection{Assumptions and Dependencies}

%Unit test Plan page
\newpage
\begin{center}
{\huge \bfseries Unit Test Plan}\\[0.5cm]
\end{center}
\section{Test Items}

\section{Functional Features to be Tested}
\begin{center}
 \begin{tabular}{||c| c| c| c||} 
 \hline
 Feature ID & RDS Source & Summary & Test Case ID \\ [0.5ex]
 1 & Implementation.PIO.BackEnd.test.Test & Passed & 01 \\
 2 & Implementation.PIO.BackEnd.test.Test & Passed & 02 \\
 3 & Implementation.PIO.BackEnd.test.Test & Passed & 03 \\
 4 & Implementation.PIO.BackEnd.test.Test & Passed & 04 \\
 5 & Implementation.PIO.BackEnd.test.Test & Passed & 05 \\
 6 & Implementation.PIO.BackEnd.test.Test & Passed & 06 \\

 \hline\hline
 %1 & 6 & 87837 & 787 \\ 
 %\hline
 %2 & 7 & 78 & 5415 \\
 %\hline
 %3 & 545 & 778 & 7507 \\
 %\hline
 %4 & 545 & 18744 & 7560 \\
 %\hline
 %5 & 88 & 788 & 6344 \\ [1ex] 
 %\hline
\end{tabular}
\end{center}

\section{Test Cases}
\subsubsection{Condition 1:AddProperty}

\subsection{Test Case 1: Add a property with negative ID}
\subsubsection{Condition 1: Back-End Server must be running}
\paragraph{Objective:}The purpose of this test is to check how the validation of add property works when invalid values are injected. 
\paragraph{Input:} Negative values are used as input , -1.
\paragraph{Outcome: } The function will throw a ArithmeticException.

\subsection{Test Case 2: Delete a property with negative ID}
\subsubsection{Condition 1: Back-End Server must be running}
\paragraph{Objective:}The purpose of this test is to check how the validation of add property works when invalid values are injected.
\paragraph{Input:} Negative values are used as input , -1.
\paragraph{Outcome: } The function will throw a ArithmeticException.

\subsection{Test Case 3: Find a property that does not exist}
\subsubsection{Condition 1: Back-End Server must be running}
\paragraph{Objective:}The purpose of this test is to test how the function handles a request when an id of a none-existing property is sent
\paragraph{Input:} Any propertyID Value that is not in the database , 33.
\paragraph{Outcome: } The function will throw a NullPointerException.

\subsection{Test Case 4: Find a profile that does not exist}
\subsubsection{Condition 1: Back-End Server must be running}
\paragraph{Objective:}The test must return an exception if the given profileID is not found in the database.
\paragraph{Input:} Any profileID Value that is not in the database , 33.
\paragraph{Outcome: } The function will throw a NullPointerException.

\subsection{Test Case 5: Delete a profile that does not exist}
\subsubsection{Condition 1: Back-End Server must be running}
\paragraph{Objective:}The test must return an exception if the given profileID to be deleted is not fond in the database.
\paragraph{Input:} Any profileID Value that is not in the database , 33
\paragraph{Outcome: } The function will throw a NullPointerException.

\subsection{Test Case 6: Delete properties that are associated with a profile that do not exist}
\subsubsection{Condition 1: Back-End Server must be running}
\paragraph{Objective:}The test must return an exception if the given profileID does not have any properties associated with it.
\paragraph{Input:} Any profileID Value that does not have any properties associated with it
\paragraph{Outcome: } The function will throw a NullPointerException.

\section{Item Pass/Fail Criteria}
Each item tested must meet a certain criteria in order to pass. These criteria are as follows:\\
Back-End
\begin{itemize}
 \item the back end server must be running
\end{itemize}
AddPropertyValidation
\begin{itemize}
 \item the property must not exist in the database
 \item the input in the parameters must be invalid
\end{itemize}
DeletepropertyValidation
\begin{itemize}
 \item the property must not exist in the database
 \item the input in the parameters must be invalid
\end{itemize}
FindProperty
\begin{itemize}
 \item the back end server must be running
 \item the property must not exist in the database
 \item the input in the parameters must be valid
\end{itemize}
FindProfile
\begin{itemize}
 \item the back end server must be running
 \item the profile must not exist in the database
 \item the input in the parameters must be valid
\end{itemize}
DeleteProfile
\begin{itemize}
 \item the back end server must be running
 \item the profile must not exist in the database
 \item the input in the parameters must be valid
\end{itemize}
RetrievePropertiesTestValidation
\begin{itemize}
 \item the back end server must be running
 \item the profile must not have any properties associated with it in the database
 \item the input in the parameters must be valid
\end{itemize}
\section{Test Deliverables}
Artefacts to be produced as part of unit testing include the following:
\begin{itemize}
\item Test Plan
\item Test Report
\item Link to Test Code
\end{itemize}

%Unit test Report page
\newpage
\begin{center}
{\huge \bfseries Unit Test Report}\\[0.5cm]
\end{center}
\section{Detailed Test Results}
\subsection{Overview of Test Results}
The test that were done were mostly on the back-end.This is where we validate our values before they are inserted into the database.all the test we did passed even those that would cause the systems fail. This was done by handling the system failure correctly,anticipating what would go wrong and handling that failure in a correct way through try and catch blocks.The tests expected the system to fail and the error to be handled ,so once it it handled in the expected way, the test pass. Junit spring framework was used for unit testing in the backend. 

\subsection{Functional Requirements Test Results}
The tests we have created for this module are contained within the files   \href{https://github.com/u13278012/IMPAKD/blob/master/Implementation/PIO/BackEnd/test/Test/PropertyTest.java}{PropertyTest} and  \href{https://github.com/u13278012/IMPAKD/blob/master/Implementation/PIO/BackEnd/test/Test/ProfileTest.java}{ProfileTest} . and \href{https://github.com/u13278012/IMPAKD/}{gitHub}.

\subsubsection{Test Case 1 (TC 4.1.1)}
The following results were obtained from the tests conducted. The tests to produce the
following results have passed/failed and the reasons are stated below.
\paragraph{Result: Pass/ Fail.}

\subsubsection{Test Case 2 (TC 4.2)}
The invalid value was handled though a try and catch block by throwing an ArithmeticException 

\paragraph{Result: Pass.}
\subsubsection{Test Case 3 (TC 4.3)}
The invalid value was handled though a try and catch block by throwing an ArithmeticException 

\paragraph{Result: Pass.}

\subsubsection{Test Case 4 (TC 4.4)}
The invalid value was handled though a try and catch block 
\paragraph{Result: Pass.}

\subsubsection{Test Case 5 (TC 4.5)}
The invalid value was handled though a try and catch block 
\paragraph{Result: Pass.}

\subsubsection{Test Case 6 (TC 4.6)}
The invalid value was handled though a try and catch block 
\paragraph{Result: Pass.}


\section{Other}
The service contracts ensure that our tests would test that the functionality would return relevant and accurate data.

The different use cases do allow for the application to be deployed into an application server because they cover all the functionality that is required of the application.
\section{Conclusions and Recommendations}
Test were done on the back-end to test for security features implemented as the application allows users to input data.Parameters are passed though the url to the back-end. So regardless of the validation we have in the font-end, malicious users might pass in  malicious code after the validation in the front-end through the url, so it is important for the application to handle both user errors and malicious code, the test have assured that the countermeasures implemented work correctly.

\end{document}

