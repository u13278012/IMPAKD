\documentclass[a4paper,12pt]{article}
\addtolength{\oddsidemargin}{-1.cm}
\addtolength{\textwidth}{2cm}
\addtolength{\topmargin}{-2cm}
\addtolength{\textheight}{3.5cm}
\newcommand{\HRule}{\rule{\linewidth}{0.5mm}}
\makeindex

\usepackage{longtable}
\usepackage[pdftex]{graphicx}
\usepackage{makeidx}
\usepackage{hyperref}
\usepackage{verbatim}
\hypersetup{
    colorlinks=true,
    linkcolor=blue,
    filecolor=magenta,      
    urlcolor=cyan,
}


% define the title
\author{IMAPKD}
\title{ Project Tender}
\begin{document}
\setlength{\parskip}{6pt}

% generates the title
\begin{titlepage}

\begin{center}
% Upper part of the page       
\includegraphics[width=1\textwidth]{./University_of_Pretoria_Logo.PNG}\\[0.4cm]  
\textsc{\LARGE COS 301}\\[0.9cm]
\textsc{\LARGE Department Of Computer Science}\\[0.3cm]

%Title

\HRule \\[0.4cm]
{ \huge \bfseries Architectural Requirements and  Initial Architecture Design Functional Requirements}\\[0.1cm]
\HRule \\[0.4cm]  

% Group Members 

\begin{minipage}{0.4\textwidth}
\begin{flushleft} \large

\emph{\Large Group Members:}\\[0.4cm]    
\emph{}\\
{\Large Diana {Obo}} \\
\emph{}\\
\emph{}\\
{\Large Priscilla {Madigoe}}\\
\emph{}\\
\emph{}\\
{\Large Kudzai {Muranga}} \\
\emph{}\\
\emph{}\\
{\Large Sandile {Khumalo}}\\
\emph{}\\
\emph{}\\

\end{flushleft}
\end{minipage}
\begin{minipage}{0.4\textwidth}
\begin{flushright} \large

\emph{ \Large Student numbers:} \\[0.4cm]  
\emph{}\\
{\Large u13134885}\\
\emph{}\\
\emph{}\\
{\Large u13049128}\\
\emph{}\\
\emph{}\\
{\Large u13278012}\\
\emph{}\\
\emph{}\\
{\Large u12031748}\\
\emph{}\\
\emph{}\\

\end{flushright}
\end{minipage}

% bottom section of title page

{\large \today}
\end{center}
\end{titlepage}
\renewcommand{\thesection}{\arabic{section}}

\newpage
\begin{center}
\textsc{\Large IMPAKD link}\\[0.5cm]
For further references see \href{https://github.com/u13278012/IMPAKD/}{gitHub}.
\today
\end{center}
\newpage
\tableofcontents{}

\newpage
%Kudzai's Section (Vision and Background) - Start 
\section{Vision}
The Property Investor Optimizer project is objective is to evaluate whether a certain rental property is worth buying. It does this by calculating the Return of Investment (ROI) of a property, which can be compared with another property's ROI, to assist a user to optimize their investment strategy according to their portfolio.\\\\
The project will assist the user by helping to answer the following questions:\begin{itemize}
	\item Given a certain bond (interest rate, deposit as a percentage of property value), rental (occupancy rate, agent commission, rental amount) and environmental conditions (Interest rate, inflation) what is the ROI?
	\item When is it better to pay a higher or lower deposit for a bond?
	\item Between two rental scenarios which provides the greater ROI?
	\item Is it better to try and pay off the bond as fast as possible by paying in extra capital?
	\item How does purchasing another property influence a user's ROI and at which point would this be a good idea?
	\item At which point does it make sense to buy another property?
	\item How much tax will the user have to pay?

\end{itemize}
\newpage
\section{Background}
The project was given to us by our client, CSIR, so that we can research how the ROI of different configurations of rental properties can answer the questions listed in the Vision section of this document. Answers to these questions can be used to help users of the system choose to buy the best property that fits their portfolio and requirements with the ease of not having to manually evaluate the property themselves. The project can also be used for property-related research.

%Kudzai's Section (Vision and Background) - Stop

\section{Scope}
\includegraphics[width=1\textwidth]{./Images/Scope.png}

\newpage
\section{Software Architecture}
\subsection{Architecture requirements}
%Kudzai's Section (Architectural Scope and Quality requirements) - Start
\subsubsection{Architectural scope}
The project will be implemented using the Model View Controller, MVC, structural pattern.
The system will consist of:\begin{itemize}
	\item A website that the user will interact with in order to use the system.
	\item A database that will be used to persist objects that need to be saved.
	\item A server that will deal with the process execution and calculations of the system.
	\item A notification system that will be used to notify the user of important information that concerns them.
\end{itemize}
\subsubsection{Quality requirements}
	\begin{itemize}
		\item{\bfseries \underline{Flexibility}}\\\\
		The system must be be able to be accessed  by more than one access channel, mainly via a computer and a 		 		smart phone. It must also not be locked to any one persistence technology. 
		\item{\bfseries \underline{Maintainability}}\\\\
		The system will be implemented with the Model View Controller structural pattern, which allows for modularisation. This 			allows the different components of the system to be maintained independently of each other. Angularjs, a 					widely-used technology with extensive support, will be used to implement MVC.
		
		\item{\bfseries \underline{Scalability}}\\\\
		The system must be able to support as many users as possible since it is a website that will be available to a very large 			audience. Enterprise Java Beans (EJB) will be used because it can be used to develop scalable and robust enterprise 			level applications.
		
		\item{\bfseries \underline{Reliability}}\\\\
		The website must have as little downtime as possible, and must display correct and accurate information at all times.
		
		\item{\bfseries \underline{Usability}}\\\\
		Users of the system must find it very easy and intuitive to use, even to users without extensive computer literacy. The 			system must be as efficient as possible. This will be done by implementing the latest trends in website user interface 			design.
		
		\item{\bfseries \underline{Performance}}\\\\
		The system must execute processes smoothly between users' actions and the website, along with the graphs and 			statistics it will provide, must respond in a reasonable amount of time.
		
		\item{\bfseries \underline{Security}}\\\\
		There must be no direct manipulation of the database and the server. Users will be required to login or register to be 			able to use the system. The database will be backed up to a CSV file.
		
		\item{\bfseries \underline{Auditability}}\\\\
		Users' activities must be logged for reporting purposes.
		
		\item{\bfseries \underline{Testability}}\\\\
		The system must be tested to ensure that each of its components are working properly. Each service contract must 			meet its pre-conditions and post-conditions. Unit and integration tests will be used to this end.
		
		\item{\bfseries \underline{Integrability}}\\\\
		All independent components of the system must be integrated together and the system must work as one.
\end{itemize}

%Kudzai's Section (Architectural Scope and Quality requirements) - End
%Priscilla's Section (Integration and Access Channel Requirements) - Start

\subsubsection{Integration and access channel requirements}
	\begin{itemize}
		\item{\bfseries \underline{Integration}}
		\begin{itemize}
           		\item Logging into the system is done over a HTTPS POST method.
			\item The user's login details are kept in an HTTP session so the user does not need to log in everytime he/she 					makes a request to the server.
			\item The HTTPS sessions are invalidated when the user terminates his/her session by logging out.
			\item Communication between the server (back-end) and the webpage (front-end) will be facilitated by the REST 					method which uses JSON objects and HTTPS methods to send requests and get responses.
			\item The "Create, Read, Update and Delete" or CRUD actions that will make changes to the databse will be 					logged automatically. This will ensure auditability of the system. 
		\end{itemize}

		\item{\bfseries \underline{Human Access Channel}}
			\begin{itemize}
				\item End-users interact with the Web client to display the required information and do desired actions. 
			\end{itemize}

		\item{\bfseries \underline{System Access Channel}}
			\begin{itemize}
				\item The Web-based component of the system will be implemented in "Angularjs" which utilizes JavaScript, 				HTML and "Handlebar.js".  
			\end{itemize}
	\end{itemize}
%Priscilla's Section - End

% Diana's Section - start 
\subsubsection{Architectural constraints}
	\begin{itemize}
		\item User
			\begin{itemize}
				\item Has to be registered and his/her details have to exist in the system in order to login and be able to use the system
			\end{itemize}
		\item Time
			\begin{itemize}
				\item If a user is logged in and remains inactive for more than 30 minutes, the user will have to login again before they can use the system again
			\end{itemize}
	\end{itemize}	
% Diana's Section - end 

% Diana's Section - start 
\subsection{Architectural patterns or styles}
 \textbf{Layer (\textit{object-oriented design})}
\begin{itemize}
\item \texttt{Services Layer}\\
The CRUD actions that will change the database will be intercepted in this layer.
\item \texttt{Business Logic Layer}\\
Will capture information of requests from objects and submit the services request.
\item\texttt{Multi-Layered}\\
MVC model uses this approach to enable the separation of business logic.
\item \texttt{Transport Layer}\\
Used to provide devices with connection to a router.
\end{itemize}
\textbf{MVC (\textit{Model View Controller})}\\
Allows the system's states to change and it encapsulates the interactions
from the user and transforms these interactions into business logic.\\
\textsc{Reason:}
	\begin{itemize}
		\item Reduce presentation layers complexity and  improves flexibility
			\begin{itemize}
				\item Separates responsibilities
					\begin{itemize}
						\item Provide view onto information -\textbf{\textit{ View}}
						\item React to user events - \textbf{\textit{Controller}}
						\item Provide business services  and data - \textbf{\textit{Model}}
					\end{itemize}
				\item Allows each component to change independently
			\end{itemize}
		\item Full decoupling
			\begin{itemize}
				\item Model from both \textit{view} and \textit{controller}
			\end{itemize}
		\item Simplification
			\begin{itemize}
				\item Through separation of concerns
			\end{itemize}
		\item Reuse
			\begin{itemize}
				\item \textit{Model} components and \textit{View} components 
			\end{itemize}
		\item Maintainability
			\begin{itemize}
				\item Different components can be used, developed and maintained by different members of a team
					\begin{itemize}
						\item \textit{Model} - backend developers
						\item \textit{View} - UI designers
						\item \textit{Controller} - Front-end developers
					\end{itemize}
			\end{itemize}
		\item Improved Testability
			\begin{itemize}
				\item Model/business services tested independently of UI
				\item UI tested with mock model
			\end{itemize}
	\end{itemize}

%Diana's Section  - end

%Sandile's Section Start
\subsection{Use of reference architectures and frameworks}
\begin{itemize}
\item Django
	\begin{itemize}
		\item Our System is going to be web based, so Angularjs is used for web development front end of the system. The structural pattern we are going to use is MVC and Angularjs implements MVC
	\end{itemize}
\item JavaEE
	\begin{itemize}
		\item JavaEE contains most of the frameworks and technologies we need to develop  and deploy our system.
	\end{itemize}
\end{itemize}
%Sandile's Section End

%Priscilla's Section (Access and Integration Channel Requirements) - Start
\subsection{Access and integration channels}
	\begin{itemize}
		\item This is a stand-alone application and therefore will not use other applications for all the required functionality.
		\item Plug-ins and APIs will be included, and will therefore be integrated with the main application to add specialised 				functionality. 
	\end{itemize}
%Priscilla's Section - End

%Sandile's Section Start
\subsection{Technologies}

\begin{itemize}
	\item \textit{HTML5 : }It will be used to create the front end of the system
\end{itemize}
\begin{itemize}
	\item \textit{Javascript :} front end to verify the log in details for each user it will keep track of the user logged in.
\end{itemize}
\begin{itemize}
	\item \textit{JPA:} It will be used to manage the relational data in JavaEE
\end{itemize}
\begin{itemize}
	\item \textit{EJBs:} It will be used to manage concurrency control in the system
\end{itemize}
\begin{itemize}
	\item \textit{Angularjs:} To implement the MVC pattern
\end{itemize}
\begin{itemize}
	\item \textit{CSS:} It will be used for the styling of the web page
\end{itemize}
\begin{itemize}
	\item \textit{Bootstrap:} It will be used for the styling of the web page 
\end{itemize}
\begin{itemize}
	\item \textit{Web browsers:} Any web browser that supports HTML 5
\end{itemize}
\begin{itemize}
	\item \textit{Postgresql:} The database that will be used to store the data
\end{itemize}

\begin{itemize}
	\item \textit{QUnits:} It is a powerful framework for javaScript unit testing, it is easy to use and It can test any generic javaScript code
\end{itemize}
%Sandile's Section End

\section{Design overview}
\includegraphics[width=1\textwidth]{./Images/Overview.png}

\newpage
\section{Functional requirements and application design}

% Diana's Section  - start
\subsection{Use case prioritization}
 \underline{\textbf{Critical:}}
	\begin{itemize}
		\item calculateROI
	\end{itemize}
\underline{\textbf{Important:}}
	\begin{itemize}
		\item Register
		\item Login
		\item logout
		\item addProperty
		\item updateProperty	
		\item deleteProperty
		\item displayGraphs
		\item displayStatistics	
	\end{itemize}
\underline{\textbf{Nice-to-have:}}
	\begin{itemize}
		\item updateProfile
		\item generateReport	
	\end{itemize}

% Diana's Section  - end

% Priscilla's Section - Start
\subsection{Use case/Services contracts}
\textbf{\large{login}}
	\begin{itemize}
		\item \textit{Pre-Conditions:}
			\begin{itemize}
				\item The user must not be logged in.
				\item The user must have already registered.
				\item The login details must match the details that exist in the database.
			\end{itemize}
		\item \textit{Post-Conditions:}
			\begin{itemize}
				\item The user will be logged into the system.
				\item The user will be able to use the system.
				\item The activity will be logged automatically.
			\end{itemize}
		\item \textit{Request and Results Data Structures:}
	\end{itemize}
\textbf{\large{register}}
	\begin{itemize}
		\item \textit{Pre-Conditions:}
			\begin{itemize}
				\item The specified username (email address) and password fields must not exist in the system.
				\item The user must not be logged in.
			\end{itemize}
		\item \textit{Post-Conditions:}
			\begin{itemize}
				\item The specified user details will be added into the system.
				\item The system will send an email to verify the email address specified.
			\end{itemize}
		\item \textit{Request and Results Data Structures:}
	\end{itemize}
\textbf{\large{logout}}
	\begin{itemize}
		\item \textit{Pre-Conditions:}
			\begin{itemize}
				\item The user must be logged in.
			\end{itemize}
		\item \textit{Post-Conditions:}
			\begin{itemize}
				\item The session must end.
				\item The user should not be logged in.
				\item The user should be redirected to the login page.
			\end{itemize}
		\item \textit{Request and Results Data Structures:}
	\end{itemize}
\textbf{\large{viewProfile}}
	\begin{itemize}
		\item \textit{Pre-Conditions:}
			\begin{itemize}
				\item The user must be logged in.
				\item The user profile must correspond with the credentials of the user logged in.
			\end{itemize}
		\item \textit{Post-Conditions:}
			\begin{itemize}
				\item The profile information requested should be returned.
			\end{itemize}
		\item \textit{Request and Results Data Structures:}
	\end{itemize}

% Priscilla's Section - End
% Diana's Section - start
\textbf{\large{updateProperty}}
	\begin{itemize}
		\item \textit{Pre-Conditions:}
			\begin{itemize}
				\item user must be logged in 
			\end{itemize}
		\item \textit{Post-Conditions:}
			\begin{itemize}
				\item the property page and associated fields must be updated
				\item database should be updated 
			\end{itemize}
		\item \textit{Request and Results Data Structures:}
	\end{itemize}
\textbf{\large{deleteProperty}}
	\begin{itemize}
		\item \textit{Pre-Conditions:}
			\begin{itemize}
				\item user must be logged in 
				\item property must exist
			\end{itemize}
		\item \textit{Post-Conditions:}
			\begin{itemize}
				\item property is deleted
				\item user must not see deleted property anymore
			\end{itemize}
		\item \textit{Request and Results Data Structures:}
	\end{itemize}
\textbf{\large{addProperty}}
	\begin{itemize}
		\item \textit{Pre-Conditions:}
			\begin{itemize}
				\item user must be logged in
				\item user must be navigated to the addProperty page 
				\item no duplicate properties
			\end{itemize}
		\item \textit{Post-Conditions:}
			\begin{itemize}
				\item property is added
				\item user should be able to view addProperty
			\end{itemize}
		\item \textit{Request and Results Data Structures:}
	\end{itemize}
% Diana's Section - end
% Kudzai's Section - Start
\textbf{\large{updateProfile}}
	\begin{itemize}
		\item \textit{Pre-Conditions:}
		\begin{itemize}
			\item user must be logged in
			\item user's profile must match with the one used to log in
		\end{itemize}
		\item \textit{Post-Conditions:}
		\begin{itemize}
			\item database is updated with edited user profile information
			\item user's profile is edited
		\end{itemize}
		\item \textit{Request and Results Data Structures:}
	\end{itemize}
\textbf{\large{compareTwoProperties}}
	\begin{itemize}
		\item \textit{Pre-Conditions:}
		\begin{itemize}
			\item user must be logged in
			\item ROI of both properties must be calculated
		\end{itemize}
		\item \textit{Post-Conditions:}
		\begin{itemize}
			\item property with greater ROI indicated
		\end{itemize}
		\item \textit{Request and Results Data Structures:}
	\end{itemize}
\textbf{\large{generateReport}}
	\begin{itemize}
		\item \textit{Pre-Conditions:}
		\begin{itemize}
			\item user must be logged in
			\item statistics for the report must have been retrived from database
		\end{itemize}
		\item \textit{Post-Conditions:}
		\begin{itemize}
			\item report generated and displayed
		\end{itemize}
		\item \textit{Request and Results Data Structures:}
	\end{itemize}
\textbf{\large{getDefaultValues}}
	\begin{itemize}
		\item \textit{Pre-Conditions:}
		\begin{itemize}
			\item user must be logged in
			\item default values must existi in the system
		\end{itemize}
		\item \textit{Post-Conditions:}
		\begin{itemize}
			\item default values are retrieved
		\end{itemize}
		\item \textit{Request and Results Data Structures:}
	\end{itemize}
% Kudzai's Section - end
\textbf{\large{displayGraphs}}
	\begin{itemize}
		\item \textit{Pre-Conditions:}
		\item \textit{Post-Conditions:}
		\item \textit{Request and Results Data Structures:}
	\end{itemize}
\textbf{\large{displayStatistics}}
	\begin{itemize}
		\item \textit{Pre-Conditions:}
		\item \textit{Post-Conditions:}
		\item \textit{Request and Results Data Structures:}
	\end{itemize}
\textbf{\large{calculateStatistics}}
	\begin{itemize}
		\item \textit{Pre-Conditions:}
		\item \textit{Post-Conditions:}
		\item \textit{Request and Results Data Structures:}
	\end{itemize}
\textbf{\large{calucalateROI}}
	\begin{itemize}
		\item \textit{Pre-Conditions:}
		\item \textit{Post-Conditions:}
		\item \textit{Request and Results Data Structures:}
	\end{itemize}

\subsection{Required functionality}
%Priscilla's Section - Start
\subsubsection{Login}
\includegraphics[width=1\textwidth]{./Images/processSpecification/Priscilla/Login.jpg}
\subsubsection{Logout}
\includegraphics[width=1\textwidth]{./Images/processSpecification/Priscilla/Logout.jpg}
\subsubsection{Register}
\includegraphics[width=1\textwidth]{./Images/processSpecification/Priscilla/Register.jpg}
%Priscilla's Setion - End

% Diana's Section - starts
\subsubsection{addProperty}
\includegraphics[width=1\textwidth]{./Images/requiredFunctionality/addProperty.png}
\subsubsection{deleteProperty}
\includegraphics[width=1\textwidth]{./Images/requiredFunctionality/deleteProperty.png}
\subsubsection{updateProperty}
\includegraphics[width=1\textwidth]{./Images/requiredFunctionality/updateProperty.png}
% Diana's Section - end
% Kudzai's Section - starts
\subsubsection{compareTwoProperties}
\includegraphics[width=1\textwidth]{./Images/requiredFunctionality/compareTwoProperties.png}
\subsubsection{generateReport}
\includegraphics[width=1\textwidth]{./Images/requiredFunctionality/generateReport.png}
\subsubsection{getDefaultValues}
\includegraphics[width=1\textwidth]{./Images/requiredFunctionality/getDefaultValues.png}
% Kudzai's Section - end

% Sandile's Section - starts
\subsubsection{Display Statistics}
\includegraphics[width=1\textwidth]{./Images/requiredFunctionality/displayStatsUseCase.jpg}
\subsubsection{Display Graph}
\includegraphics[width=1\textwidth]{./Images/requiredFunctionality/displayGraphRequestUseCase.jpg}

% Sandile's Section - end

\subsection{Process specifications}
%Priscilla's Section - Start
\subsubsection{Login}
\includegraphics[width=1\textwidth]{./Images/processSpecification/Priscilla/loginRequest.jpg}
\subsubsection{Logout}
\includegraphics[width=1\textwidth]{./Images/processSpecification/Priscilla/logoutRequest.jpg}
\subsubsection{Register}
\includegraphics[width=1\textwidth]{./Images/processSpecification/Priscilla/registerRequest.jpg}
%Priscilla's Section - End
\subsubsection{addProperty}
\includegraphics[width=1\textwidth]{./Images/processSpecification/addProperty.png}
\subsubsection{deleteProperty}
\includegraphics[width=1\textwidth]{./Images/processSpecification/deleteProperty.png}
\subsubsection{updateProperty}
\includegraphics[width=1\textwidth]{./Images/processSpecification/updateProperty.png}
% Kudzai's Section - starts
\subsubsection{compareTwoProperties}
\includegraphics[width=1\textwidth]{./Images/processSpecification/compareTwoProperties.png}
\subsubsection{generateReport}
\includegraphics[width=1\textwidth]{./Images/processSpecification/generateReport.png}
\subsubsection{setDefaultValue}
\includegraphics[width=1\textwidth]{./Images/processSpecification/setDefaultValue.png}
% Kudzai's Section - end

% Sandile's Section - starts
\subsubsection{Display Statistics}
\includegraphics[width=1\textwidth]{./Images/processSpecification/displayStatsActivity.jpg}
\subsubsection{displayGraphRequest}
\includegraphics[width=1\textwidth]{./Images/processSpecification/displayGraphRequestActivity.jpg}

% Sandile's Section - end

\subsection{Domain Model}
\includegraphics[width=1\textwidth]{./domainModel/domain_Model.PNG}


\newpage
\section{Open Issues}



\end{document}

